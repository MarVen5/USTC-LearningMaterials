\documentclass[UTF8]{ctexart}
\usepackage{graphicx}
\usepackage{amsmath}
\usepackage{amssymb}
\usepackage{bm}
\usepackage{float}
\usepackage{fancyhdr}
\usepackage{geometry}
\usepackage{multirow}
\usepackage{multicol}
\usepackage{subfigure}
\pagestyle{fancy}
\fancyhead{}
\renewcommand*{\headrulewidth}{0pt}
\geometry{left=2cm}
\geometry{right=2cm}
\geometry{top=2cm}
\begin{document}
\title{\songti{\textbf{大学物理实验报告——基于彩色 CCD 的棱镜摄谱实验}}}
\author{\songti{\large{}}}
\maketitle
\pagenumbering{arabic}
\paragraph*{实验背景:}~\\
\indent 光谱学是研究各种物质的光谱的产生及其同物质之间相互作用。光谱是电磁
波辐射按照波长的有序排列;通过光谱的研究,人们可以得到原子、分子等的能
级结构、电子组态、化学键的性质、反应动力学等多方面物质结构的知识。光谱学
也为化学分析提供了重要的定性与定量的分析方法。发射光谱可以分为三种不同类
别的光谱:线状光谱、带状光谱、连续光谱。线状光谱主要产生于原子,带状光
谱主要产生于分子,连续光谱则主要产生于白炽的固体或气体放电。~\\

\paragraph*{实验目的:}~\\
(1)了解能将复色光变成单色光的一系列分光元件,掌握它们的原理;~\\
(2)学习用插值法计算分光后得到的未知光谱。~\\

\paragraph*{实验原理:}~\\
\indent 棱镜摄谱仪是利用棱镜作为分光元件的摄谱仪器。棱镜的分光原理如图 1 所示。
本次实验所用的是可见光范围内的小型棱镜摄谱仪,如图 2 所示。S 为
光源, L 为聚光透镜,使 S 发出的发散光会聚后均匀照亮狭缝,$\rm{S_1}$ 为狭缝,以
控制入射光的宽度;\textbf{$\rm{L_1}$ 为准直透镜,和 $\rm{S_1}$ 的距离大小等于其焦距,产生平行光}
后,均匀的照射在阿贝棱镜的入射面上,这是摄谱仪的第一部分。经透镜 $L_1$ 照
射过来的平行光,通过阿贝棱镜中的两个 30 度三棱镜分光,并作 90 度转向后出
射,经阿贝棱镜分光后的各种单色光不再相互平行,而是之间有相互较小的夹角,
这就是分光的本质所在。至此完成了摄谱仪的第二部分。经过分光后的各种单色
光,由会聚透镜 $\rm{L_2}$,将各种分离的单色光会聚成单一谱线,成像于 $\rm{L_2}$ 的谱平面
上。将彩色 CCD 的成像面置于 $\rm{L_2}$ 的谱平面上;通过彩色 CCD 连接到计算机和
显示器,可以看到各种分离的彩色谱线。
\begin{figure}[H]
    \centering
    \includegraphics[width=0.45\textwidth]{}
    \caption{棱镜分光原理}
 \end{figure}~\\
 \begin{figure}[H]
    \centering
    \includegraphics[width=0.4\textwidth]{}
    \caption{棱镜摄谱仪工作原理}
 \end{figure}~\\
\indent 利用 CCD 拍摄不同的光源谱线,记录
成图片格式存储。计算机对这些图片进行比对,用插值法,从已知谱线(氦灯谱)和未知谱
线(钠灯谱和汞灯谱)的位置(像素)关系上,就可以计算出未知谱线。如图 3,
将 $\lambda_2-\lambda_1$ 与 $d=l_2-l_1$ 近似看
成线性关系,则
\begin{equation}
    \begin{split}
        \frac{\lambda_x-\lambda_1}{\lambda_2-\lambda_1}=\frac{d_x}{d}\Rightarrow \lambda_x=\lambda_1+(\lambda_2-\lambda_1)\frac{d_x}{d}     
    \end{split}
\end{equation}
 \begin{figure}[H]
    \centering
    \includegraphics[width=0.35\textwidth]{}
    \caption{插值法计算波长原理}
 \end{figure}~\\
\indent 同时,我们在 CCD 的成像面的位置上,安装了平板玻璃视窗转换机构,
用 20 倍放大的广角目镜,可以观察到全景的彩色分离光谱。~\\

\paragraph*{实验仪器和参数:}氦灯、钠灯、汞灯、电源、狭缝、聚光透镜 L、准直透镜 $\rm{L}_1$、
阿贝复合棱镜、会聚透镜 $\rm{L}_2$、彩色 CCD、计算机。其中最重要的元件是阿贝复合
棱镜和彩色 CCD,以下是两个元件的简介。~\\
\indent 1、 阿贝复合棱镜~\\
\indent 小型摄谱仪常选用阿贝 (Abbe) 复合棱镜,它由两个 $30^{\circ}$ 角折射棱镜和一个
$45^{\circ}$ 角全反射棱镜组成,如图 4 所示。在摄谱仪中,
棱镜的主要作用是用来分光,即利用棱镜对不同波长的光有不同折射率的性质来
进行分光。折射率 $n$ 与光的波长 $\lambda$ 有关。当一束白光或其它非单色光入射到棱镜
时,由于折射率不同,不同波长(颜色)的光具有不同的偏向角,从而出射光线
方向不同。通常棱镜的折射率 $n$ 是随波长 $\lambda$ 的减小而增加的(正常色散),所以
可见光中紫光偏折最大,红光偏折最小。一般的棱镜摄谱仪都是利用这种分光作
用制成的。~\\
\indent 2、 CCD ~\\
\indent CCD 是电荷耦合器件的简称(Charge Coupled Device),也称为 CCD 图像传
感器。它使用一种高感光度的半导体材料制成,能把光线转变成电荷,通过模
数转换器芯片转换成数字信号,数字信号经过压缩以后由相机内部的闪速存储器
或内置硬盘卡保存,因而可以轻而易举地把数据传输给计算机,并借助计算机
的处理手段,根据需要来修改图像。~\\

\paragraph*{实验内容与步骤:}~\\
\indent(1) 调节光源、会聚透镜 L、狭缝中心处于等高共轴状态:
用钢板尺测量狭缝中心、会聚透镜中心、光源中心,使之等高;同时,三者必须
置于同一条直线上。打开氦灯源 5 分钟后,使光源经会聚透镜 L 在狭缝处成一个
缩小的实像并照亮狭缝。~\\
\indent(2) 调节狭缝:可调狭缝用来限制入射光束的宽窄,它也是光谱线的宽度
调节机构。狭缝的大小通过狭缝上端的手轮调整,一般以 0.1mm 为宜,转动手
轮时一定要用力均匀,轻柔。~\\
\indent(3) 观察谱线:光源通过棱镜分光后,光谱成像在观察平板玻璃处,调节会
聚透镜 $\rm{L_2}$ 的调节旋钮,使用目镜观察谱线直到清晰为止。~\\
\indent(4) 从左到右记录观察到的谱线颜色、条数、强弱。~\\
\indent(5) 更换光源,重复上述步骤。~\\
\indent(6) 摄谱:~\\
\indent a) 将观察平板玻璃转换机构向上翻转,调整支撑 CCD 的精密三维平台至合
适位置,打开 CCD 遮光罩,将 CCD 成像面居中对准光谱平面位置。
连接好 CCD,用 Toupview 软件观察到图像,此时图像可能不清晰,通过调节 CCD 三维台的 X、Y、Z 旋钮,直到
观察到的图像处于最清晰的位置。~\\
\indent b) 将氦灯光源置于光路中,按步骤 a) 调整,先将三维平台由左向右移动,观察
全部谱线。由于光谱线宽度比 CCD 靶面要宽,先拍摄氦灯光源
可见光长波段(红、黄)的光谱谱线图片,拍摄 1 次。CCD 位置保持不动,
换上汞灯光源并调整,在 CCD 上出现清晰图像;拍摄 1 次。再换上钠灯光
源,重复上述步骤。拍摄完图像后,按照具体拍摄的光源谱线对图像进行命
名保存。(如:氦-左 1,汞-左 1 等)~\\
\indent c) 移动CCD至光谱的中间位置,重复上述过程;移动CCD至光谱的右边位置,
重复上述过程。~\\
\indent(7) 光谱分析:~\\
\indent 在 PowerPoint 中打开保存好的图片,将
氦灯光源的三幅图片按照特征线重合的原则拼接在一起,同理拼接汞、氢光源的
三幅图片,这样 9 幅图片被拼接成三张完整的光源谱。然后将三种光源谱线图
依次从上往下排列,形成一张图片。再从这张谱线图上读出并记录谱线的坐标值
(像素点位置),得到坐标之后,利用插值法即通过已知波长计算未知谱线波
长。~\\

\paragraph*{实验数据与不确定度分析:}~\\
\indent 实验中拍摄到的图片如下:
\begin{figure}[H]
    \centering
    \includegraphics[width=0.37\textwidth]{}
    \caption{拼合得到的光谱(已做反色处理,从上到下依次为氦灯谱、钠灯谱和汞灯谱)}
 \end{figure}~\\
 图中氦灯谱从左到右几条谱线的波长如下(颜色均指反色处理前的颜色):
\begin{table}[H]
    \centering
    \caption{氦灯谱各条谱线波长}
    \begin{tabular}{ccccccccc}
        \hline
     编号& 1  & 2  & 3   & 4  & 5   & 6   & 7 & 8  \\
    \hline
   颜色 &红(暗) &红 &黄 &绿(淡)  &绿  &蓝绿  &蓝  &紫 \\
   波长(nm) &706.52 &667.82 &587.56 &504.77  &501.57  &492.19  &471.31  &447.15\\
    像素位置 &41164 &98175 &260170 &515171 &526129 &568163 &620170 &817169\\
        \hline                  
    \end{tabular}
    \end{table}~\\
 对钠黄光,像素位置为 252490,用上表 2、3 谱线作为参考谱线,可得 
    $$\lambda_{\mbox{\zihao{7}钠-黄}}=\lambda_2+\frac{252490-98175}{260170-98175}(\lambda_3-\lambda_2)=591.36\rm{nm} $$
    对汞双黄线左线,像素位置为 277822,用上表 2、3 谱线作为参考谱线,可得 
    $$\lambda_{\mbox{\zihao{7}汞-黄1}}=\lambda_2+\frac{277822-98175}{260170-98175}(\lambda_3-\lambda_2)=578.81\rm{nm} $$
    对汞双黄线右线,像素位置为 282824,用上表 2、3 谱线作为参考谱线,可得 
    $$\lambda_{\mbox{\zihao{7}汞-黄2}}=\lambda_2+\frac{282824-98175}{260170-98175}(\lambda_3-\lambda_2)=576.34\rm{nm} $$
    对汞绿光,像素位置为 378819,用上表 3、4 谱线作为参考谱线,可得 
    $$\lambda_{\mbox{\zihao{7}汞-绿}}=\lambda_3+\frac{378819-260170}{515171-260170}(\lambda_4-\lambda_3)=547.03\rm{nm} $$
    对汞蓝光,像素位置为 897823,用上表 7、8 谱线作为参考谱线,可得 
    $$\lambda_{\mbox{\zihao{7}汞-蓝}}=\lambda_7+\frac{897823-620170}{817169-620170}(\lambda_8-\lambda_7)=433.78\rm{nm} $$
另外,笔者采用实验室电脑的软件直接分析得波长如下:
\begin{figure}[H]
    \centering
    \includegraphics[width=0.6\textwidth]{}
    \caption{软件分析结果(已做反色处理)}
 \end{figure}~\\
\paragraph*{分析与讨论:}~\\
\indent 本实验测得的数据与参考值的对比结果如下:
\begin{table}[H]
    \centering
    \caption{测量结果与参考值的比较}
    \begin{tabular}{cccccc}
        \hline
     谱线名称(与上一节的公式部分一致)& 钠-黄  & 汞-黄1  & 汞-黄2   & 汞-绿  & 汞-蓝\\
    \hline
   参考波长(nm) &589.29 &579.07 &576.96 &546.07 &435.84\\
   测得波长(nm) &591.36 &578.81 &576.34 &547.03 &433.78\\
   误差 &0.35\% &0.04\% &0.11\% &0.18\% &0.27\%\\
    \hline    
    \end{tabular}
    \end{table}~\\
由此可见,本实验误差较小。误差可能主要来源于: 1.特征线有一定宽度,拼接图片时难以做到使之完全重合; 2.
谱线实际拍摄出来略有弯曲,且有一定宽度,导致测量谱线位置时像素点的选取不唯一; 3.光谱成像位置自身的误差。 ~\\
\indent 本实验可能的改进方案:将仅仅基于两条参考谱线以计算单条谱线波长的插值法改为线性拟合法,即根据氦灯各谱线的位置和已知
波长,用最小二乘法建立一维的位置-波长线性关系,对于未知光谱,可利用这一关系求出波长。~\\
\indent 本实验代表的摄谱方法具有实际意义。如对行星研究中可利用摄谱仪把望远镜观测到的光分解成各种波长,
所得到的不同波长的色带就是记录在照相底版上的光谱。由于行星大气层中特定气体往往吸收特定波长的光,因此吸收光谱中
的暗线可以指示特定气体的存在,这种方法甚至可以作为探测生命的手段之一。~\\

\paragraph*{思考题:}~\\
1. 实验中影响光谱清晰度的调节机构有哪些?~\\
答:(沿着光路)依次有聚光透镜 L、狭缝、准直透镜 $\rm{L}_1$、会聚透镜 $\rm{L}_2$。~\\
2. 实验中,CCD 靶面的横向宽度小于光谱成像面的横向宽度,实验中是如何完成的?~\\
答:将 CCD 的三维平台自左向右移动,依次拍摄可见光区从长波段到短波段的三个位置的图像。~\\
3. 本实验中,能否将光谱成像面的横向宽度做到小于或等于 CCD 的靶面横向宽
度?如果能,怎么做?实际实验中未做,可能的原因是什么?~\\
答:理论上可行;方法是改变会聚透镜 $\rm{L}_2$ 的材质以减小其放大率,同时调整成像面(观察平板玻璃)与 $\rm{L}_2$ 的距离以使得仍在观察平板处成像;
实际未做可能是因为若光谱成像面的横向宽度减小,成像的位置误差增大,测量位置时误差也增大。~\\
4. 三棱镜可以作为分光元件的原因是什么?~\\
答:三棱镜的折射率 $n$ 与光的波长 $\lambda$ 有关。当一束白光或其它非单色光入射到棱镜
时,由于折射率不同,不同波长(颜色)的光具有不同的偏向角,从而出射光线方向不同。
\clearpage 

\label{unknown}
\end{document}