\documentclass{article}
\usepackage[margin=0.8in]{geometry}
\usepackage{amsmath}
\usepackage{ctex}
\usepackage{siunitx}
\usepackage{multirow}
\usepackage{bigstrut}
\usepackage{graphicx}
\usepackage{hyperref} 
\title{分光计实验报告}
\author{姓名:宋建宏\,\, 学号:PB21020677\,\, 班级:203院22级5班\\ 日期:2023年5月5日}
\date{}

\begin{document}
\maketitle

\section*{实验目的}
初步了解分光计的工作原理, 学会调节、 使用分光计。进行三棱镜顶角和最小偏向角的测量, 进而测
量三棱镜材料的折射率。

\section*{实验原理}
用最小偏向角法测三棱镜材料的折射率。一束单色光以 $i_1$ 角入射到 $A B$ 面上, 经棱镜两 次折射后从 $A C$ 面射出,出射角为 $i_2^{\prime}$ 人入射光与出射光之间的夹角 $\delta$ 称为偏向角。当棱镜顶角 $A$ 一定时,当 $i_1=i_2^{\prime}$ 时, $\delta$ 为最小,称为最小偏向角,记作 $\delta_{\min }$ 。
此时有 $i_1^{\prime}=\frac{A}{2}, i_1=\frac{\delta_{\min }+A}{2}$ 。设棱镜折射率为 $n$, 则有
$$
n=\frac{\sin i_1}{\sin i_1^{\prime}}=\frac{\sin \frac{\delta_{\min }+A}{2}}{\sin \frac{A}{2}}
$$
由此可知, 要测得折射率 $n$, 需测得顶角 $A$ 和最小偏向角 $\delta_{\min }$ 。


\section*{实验仪器}
\

分光计、双面平面镜、三棱镜、汞灯、遮光板。
\section*{测量记录}
各实验测量数据及计算结果如下:

% Table generated by Excel2LaTeX from sheet 'Sheet1'
\begin{table}[htbp]
    \centering
    \caption{顶角}
      \begin{tabular}{|r|r|r|r|r|r|}
      \hline
      \multicolumn{1}{|l|}{次序} & $\theta_1$     & $\theta_2$     & \multicolumn{1}{r|}{$\theta_1'$} & \multicolumn{1}{r|}{$\theta_2'$} & \multicolumn{1}{r|}{A} \bigstrut\\
      \hline
      1     & 80°12' & 260°10' & 200°10' & 20°08' & 60°02' \bigstrut\\
      \hline
      2     & 198°49' & 18°48' & 78°50' & 258°47' & 60°00' \bigstrut\\
      \hline
      3     & 199°46' & 19°46' & 319°45' & 139°45' & 60°00' \bigstrut\\
      \hline
      \end{tabular}%

  \end{table}%
  

  
% Table generated by Excel2LaTeX from sheet 'Sheet1'
\begin{table}[htbp]
    \centering
    \caption{最小偏向角}
      \begin{tabular}{|r|r|r|r|r|r|}
      \hline
      \multicolumn{1}{|l|}{次序} & $\theta_1$     & $\theta_2$     & $\theta_1'$    & $\theta_2'$    & $\delta_{min}$ \bigstrut\\
      \hline
      1     & 339°00' & 159°02' & 30°40' & 210°41' & 50°40' \bigstrut\\
      \hline
      2     & 254°08' & 74°10' & 305°47' & 125°48' & 51°38' \bigstrut\\
      \hline
      3     & 257°19' & 77°21' & 308°58' & 128°59' & 51°38' \bigstrut\\
      \hline
      \end{tabular}%

  \end{table}%
  

\section*{数据处理}
为计算方便,在处理过程中使用十进制。
\subsection*{顶角}
平均值
$$
\bar{A}=\frac{60.0333\mathrm{^{\circ}}+60.0000\mathrm{^{\circ}}+60.0000\mathrm{^{\circ}}}{3}\approx60.0111\,\mathrm{^{\circ}}
$$
标准差
$$
\begin{aligned}
\sigma&=\sqrt{\frac{(60.0333\mathrm{^{\circ}}-60.0111\mathrm{^{\circ}})^2+(60.0000\mathrm{^{\circ}}-60.0111\mathrm{^{\circ}})^2+(60.0000\mathrm{^{\circ}}-60.0111\mathrm{^{\circ}})^2}{3-1}}\\
&\approx0.0192\mathrm{^{\circ}}
\end{aligned}
$$
A类不确定度
\[U_{A,A}=t_{0.95}\frac{\sigma}{C}=4.3\times\frac{0.0192}{\sqrt{3}}\approx0.0477\,\mathrm{^{\circ}}\]
B类不确定度
$$
U_{A,B}=\frac{k_{0.95}}{C}\sqrt{\Delta_\text{仪}^2+\Delta_\text{估}^2}=\frac{1.645}{\sqrt{3}}\times\sqrt{(0.0167\mathrm{^{\circ}})^2+(0.0083\mathrm{^{\circ}})^2}\approx0.0177\,\mathrm{^{\circ}}
$$
故顶角不确定度为
$$
\begin{aligned}
U_A&=\sqrt{U_{A,A}^2+U_{A,B}^2}=\sqrt{0.0477^2+0.0177^2}\approx0.0509\,\mathrm{^{\circ}}\approx3'\quad(P=0.95)
\end{aligned}
$$


\subsection*{最小偏向角}

平均值
$$
\bar{\delta}_{\min}=\frac{50.6667+51.6333+51.6333}{3}\,\mathrm{^{\circ}}=51.3111\,\mathrm{^{\circ}}
$$
标准差
$$
\begin{aligned}
\sigma&=\sqrt{\frac{(50.6667-51.3111)^2+(51.6333-51.3111)^2+(51.6333-51.3111)^2}{3-1}}\,\mathrm{^{\circ}}\\
&=0.5581\,\mathrm{^{\circ}}
\end{aligned}
$$
A类不确定度
\[U_{\delta_{\min},A}=t_{0.95}\frac{\sigma}{C}=4.3\times\frac{0.5581\,\mathrm{^{\circ}}}{\sqrt{3}}\approx1.386\,\mathrm{^{\circ}}\]
B类不确定度
$$
U_{{\delta}_{\min},B}=\frac{k_{0.95}}{C}\sqrt{\Delta_\text{仪}^2+\Delta_\text{估}^2}=\frac{1.645}{\sqrt{3}}\times\sqrt{(0.0167\mathrm{^{\circ}})^2+(0.0083\mathrm{^{\circ}})^2}\approx0.0177\,\mathrm{^{\circ}}
$$
故最小偏向角的不确定度为
$$
\begin{aligned}
U_{{\delta_{\min}}}=\sqrt{U_{\delta_{\min},A}^2+U_{\delta_{\min},B}^2}=\sqrt{1.386^2+0.0177^2}\approx1.386\,\mathrm{^{\circ}}\approx\,80'\quad(P=0.95)
\end{aligned}
$$
\subsection*{折射率}
绿光下玻璃三棱镜折射率n
$$
n=\frac{\sin{\left(\frac{A}{2} + \frac{\delta_{\min}}{2} \right)}}{\sin{\left(\frac{A}{2} \right)}}=\frac{\sin{\left(\frac{60.0111}{2}+\frac{51.3111}{2}\right)}}{\sin{\left(\frac{60.0111}{2}\right)}}\,\mathrm{}\approx1.65115\,\mathrm{}
$$
折射率n的不确定度
$$
\begin{aligned}
U_{n}&=\sqrt{\left(\frac{\partial n}{\partial A}U_{A}\right)^2+\left(\frac{\partial n}{\partial \delta_{\min}}U_{\delta_{\min}}\right)^2}\\
&=\sqrt{\left(\left(\frac{\cos{\left(\frac{A}{2} + \frac{\delta_{\min}}{2} \right)}}{2 \sin{\left(\frac{A}{2} \right)}} - \frac{\sin{\left(\frac{A}{2} + \frac{\delta_{\min}}{2} \right)} \cos{\left(\frac{A}{2} \right)}}{2 \sin^{2}{\left(\frac{A}{2} \right)}}\right)U_{A}\right)^2+\left(\frac{\cos{\left(\frac{A}{2} + \frac{\delta_{\min}}{2} \right)}}{2 \sin{\left(\frac{A}{2} \right)}}U_{\delta_{\min}}\right)^2}\\
&\approx0.014\,\mathrm{},P=0.95
\end{aligned}
$$
绿光下玻璃三棱镜折射率n最终结果
$$
n=1.651\pm0.014\quad (P=0.95) 
$$
\subsection*{误差分析}
本次测量所得折射率不确定度较大,审查数据发现原因在于最小偏向角测量值相差较大,应为操作失误。

将最小偏向角的偏离较大数据剔除后重新计算得 \[U_n\approx7.85\times 10^{-4}\]\[n=1.6543\pm 0.0008\]


\section*{思考题}
1.已调好望远镜光轴垂直主轴, 若将平面镜取下后, 又放到载物台上(放的位置与拿下
前的位置不同) , 发现两镜面又不垂直望远镜光轴了, 这是为什么? 是否说明望远镜光轴还
没调好?

答:这是因为载物台没有调整至水平。并不能说明望远镜光轴还没调好, 只有将载物台先调水平后才能确认望远镜光轴
受否已经调好。

\end{document}