\documentclass[UTF8]{ctexart}
\usepackage{ntheorem}
\usepackage{amsmath}
\usepackage{upgreek}
\usepackage{graphicx}
\usepackage{footmisc}
\usepackage{amssymb}
\usepackage{makecell}
\title{定向耦合器及圆柱谐振腔参数测量实验}
\author{PB20051086 金秋实}
\usepackage{geometry}
\geometry{a4paper,scale=0.75}
\date{\today}
\begin{document}
\maketitle
\section{实验目的}
(1)了解定向耦合器和圆柱谐振腔的结构和基本原理;

(2)掌握定向耦合器和圆柱谐振腔特性参数的测量方法。 
\section{实验原理}
(1)定向耦合器

定向耦合器是一种具有方向性的功率耦合(分配)元件,它是一个四端口 元件。如图4所示,“①→②”是一条传输系统,称为主传输线(主线);“③→④”为另一条传输系统,称为副传输线(副线),主、副线之间通过耦合机构(例如缝隙、孔、分支线、耦合线段等)把主线功率的一部分(或全部)耦合到副线中去。而且要求功率在副线中只传向某一输出端口,另一端口则无输出。即是说,功率的耦合(分配)是有方向性的,因此称为定向耦合器。
\begin{figure}[!htp]
    \centering
    \includegraphics{图片1.jpg}
    \caption{定向耦合器的原理图}
\end{figure}

设图1中端口“①”为输入端,端口“②”为直通输出端,端口“③”为耦合输 出端,端口“④”为隔离端。描述定向耦合器的性能指标有:耦合度、隔离度、方向性。

设定向耦合器的散射矩阵为:
\begin{equation}
    [S] = 
    \begin{bmatrix}
        S_{11} & S_{12} & S_{13} & S_{14} \\
        S_{21} & S_{22} & S_{23} & S_{24} \\
        S_{31} & S_{32} & S_{33} & S_{34} \\
        S_{41} & S_{42} & S_{43} & S_{44}
    \end{bmatrix}
\end{equation}

(a)输入端“①”的输入功率P1与耦合端“③”的输出功率P3之比定义为耦合度,记作C。
\begin{equation}
    C = 10lg_{10}\frac{P_1}{P_3}= 20 lg \frac{1}{|S_{31}|}
\end{equation}

(b)输入端“①”的输入功率 P1和隔离端“④”的输出功率P4之比定义为隔离度,记作I。
\begin{equation}
    I = 10 lg\frac{P_1}{P_4} = 20 lg \frac{1}{|S_{41}|}
\end{equation}

(c)在理想情况下,副线中耦合口输出时,隔离口应无输出,但实际上由于 设计公式的近似或工艺制造上的原因,常有一些输出,难以达到理想隔离。因此把耦合端“③”的输出功率P3与隔离端“④”的输出功率P4之比定义为方向性系,记作D。
\begin{equation}
    D = 10 lg \frac{P_3}{P_4} = 20 lg \left|\frac{S_{31}}{S_{41}}\right| = I - C
\end{equation}

其常用谐振模有$TM_{010}$、$TE_{111}$、$TE_{011}$三种,在$l<2.1a$ 时,圆柱谐振腔的最低模为:$TM_{010}$,在$l>2.1a$时,圆柱谐振腔的最低模为:$TE_{111}$,$TE_{011}$在圆柱谐振腔中不是最低模,但由于该模式只有沿$\phi$方向的臂电流分布,损耗很小,故其品质因数最高,因而多用该模式做精度很高的稳频腔或波长计。描述谐振器的参量有谐振频率$f_0$、品质因数Q。
\begin{figure}[!htp]
    \centering
    \includegraphics{图片2.png}
    \caption{圆柱谐振腔示意图}
\end{figure}
矢量网络分析仪测量圆柱谐振腔的谐振频率和品质系数主要原理是:当谐振时,圆柱谐振腔的功率最大,此时对应的频率为谐振频率;当圆柱谐振腔的功率降低为原来一半时,此时带宽为半功率带宽,也被称为3dB带宽,假设谐振频率为$f_0$,两个半功率频率点分别为:$f_1$和$f_2$ ($f_2>f_1$ ),那么品质系数为:
\begin{equation}
    Q = \frac{f_0}{f_2 - f_1}
\end{equation}

\section{实验内容}
(1) 测量定向耦合器的耦合度和隔离度;

(2) 测量圆柱谐振腔的谐振频率和品质系数。

\section{实验装置}

该传输线微波参数测试系统主要包含产品如下表所示:
\begin{table}[!htp]
    \centering
    \caption{系统组成表}
    \begin{tabular}{|c| c |c| c| c|}
        \hline
        序号 & 名称 & 型号 & 数量 & 备注 \\
        \hline
        1 & 波导高方向性耦合器 & HD-32WC10N & 1 & \\\hline
        2 & 波导固定衰减器 & HD-32WFA20 & 1 & \\\hline
        3 & 匹配负载 & HD-32WL1.03 & 3 & \\\hline
        4 & 波导短路板 & HD-32WS & 1 & \\\hline
        5 & 波导同轴转换 & HD-32WCAN & 2 & \\\hline
        6 & 波导升降平台 & HD-SJPT-T1 & 5 & \\\hline
        7 & 电缆 & HD-18DLB50NJJ1500 & 2 & \\\hline
        8 & 被测件 定向耦合器& HD-32WC10N& 1&\\\hline
        & 被测件 圆柱谐振腔& HD-32XZQ3&&\\\hline
        9& 定位螺钉&&50&\\\hline
        10& 矢量网络分析仪&FPC1500&1&\\
        \hline
        \end{tabular}
\end{table}
\section{实验步骤}
(1)定向耦合器参数测量

设定矢量网络分析仪的频率范围为2.7~3GHz;

(a) 回波损耗测量

反射定标校准:

按照图3进行反射定标,在被测件的连接处,用短路板替换被测件直波导进行连接,进行反射定标,在S21测试模式下,采用S21 Normalize校准,使得回波损耗RL=0dB;

回波损耗测量:

定标完成后,去掉短路板,将被测件定向耦合器一端与波导高方向性耦合器相连,另外几端匹配负载,分别测量2.7GHz,2.75GHz,2.8GHz,2.85GHz,2.9GHz,2.95GHz,3GHz下的回波损耗,每个频率测量三次,取平均值。根据回波损耗,计算反射参数。

按照上述方法分别其他端口的回波损耗。
\begin{figure}[!htp]
    \centering
    \includegraphics{图片3.png}
    \caption{定向耦合器回波损耗测量示意图}
\end{figure}

(b) 耦合度和隔离度测量

传输定标:按照图4进行连接,将被测件用同轴转换替代,在S21测试模式下,采用S21 Normalize校准,使得回波损耗RL=0dB,完成传输定标;

耦合度测量:按照图4进行连接,被测件耦合器1端口与高方向性耦合器相连,3端口与矢量网络分析仪相连,分别测量2.7GHz,2.75GHz,2.8GHz,2.85GHz,2.9GHz,2.95GHz,3GHz下的回波损耗,每个频率测量三次,取平均值。所测数值即为被测件耦合器的耦合度。

隔离度测量:按照图4进行连接,被测件耦合器2端口与高方向性耦合器相连,3端口与矢量网络分析仪相连,分别测量2.7GHz,2.75GHz,2.8GHz,2.85GHz,2.9GHz,2.95GHz,3GHz下的回波损耗,每个频率测量三次,取平均值。所测数值即为被测件耦合器的隔离度。
\begin{figure}[!htp]
    \centering
    \includegraphics{图片4.png}
    \caption{定向耦合器耦合度及隔离度测量示意图}
\end{figure}

(2)圆柱谐振腔参数测量

设定矢量网络分析仪的频率范围为2.7~3GHz;

(a) 利用反射参数测量谐振频率

反射定标校准:

将所需器件按照图5方式进行连接,在波导高方向性耦合器右端采用短路板进行反射定标,矢量网络分析仪显示回波损耗 RL=0dB;

利用回波损耗测量谐振频率:

将被测件谐振腔一端口与波导高方向性耦合器相连,另外一端接波导匹配负载,分别调节谐振腔的刻度为0mm,10mm,20mm(初始长度,即0刻度对应长度为58.5mm,谐振腔半径为96mm),记录最小回波损耗对应的频率,即为谐振频率。
\begin{figure}[!htp]
    \centering
    \includegraphics{图片6.png}
    \caption{圆柱谐振腔(图中直波导)反射参量测量示意图}
\end{figure}

(b) 利用传输参数测量谐振频率及品质系数

传输定标校准:

将所需器件按照图6进行连接,将固定衰减器的右侧端口与高方向耦合器的左侧端口相连接,用传输法进行定标,仪表显示RL=0dB

利用传输参量测量谐振频率:

将被测件谐振腔一端口与波导高方向性耦合器相连,另外一端接波导衰减器,分别调节谐振腔的刻度为0mm,10mm,20mm(初始长度,即0刻度对应长度为58.5mm,谐振腔半径为96mm),记录最小回波损耗对应的频率,即为谐振频率。根据3dB原则,分别测量3dB对应的频宽,并根据此计算品质系数。
\begin{figure}[!htp]
    \centering
    \includegraphics{图片5.png}
    \caption{圆柱谐振腔(图中直波导)传输参量测量示意图}
\end{figure}

1.6实验数据处理

(1)根据反射回波损耗,计算被测件定向耦合器三端口的反射参量(模),绘制反射参量随频率的变化曲线;
\begin{figure}[!htp]
    \centering
    \includegraphics[scale = 0.5]{反射参量.png}
    \caption{反射参量随频率变化曲线}
\end{figure}

(2)根据被测件定向耦合器的耦合度和隔离度,计算方向性,并绘制耦合度、隔离度、方向性随频率的变化曲线;
\begin{figure}[!htp]
    \centering
    \includegraphics[scale = 0.5]{三度.png}
    \caption{耦合度、隔离度、方向性随频率变化曲线}
\end{figure}

(3)根据谐振频率,计算不同长度下的圆柱谐振腔的品质系数。
\begin{figure}[!htp]
    \centering
    \includegraphics[scale = 0.5]{品质因子.png}
    \caption{不同长度下的品质系数}
\end{figure}

\section{实验数据}
实验原始数据如下图:
\begin{figure}[!htp]
    \centering
    \includegraphics[scale = 0.1]{实验数据1.jpg}\qquad
    \includegraphics[scale = 0.1]{实验数据2.jpg}\qquad
    \includegraphics[scale = 0.05]{实验数据3.jpg}
    \caption{实验原始数据}
\end{figure}
\end{document}