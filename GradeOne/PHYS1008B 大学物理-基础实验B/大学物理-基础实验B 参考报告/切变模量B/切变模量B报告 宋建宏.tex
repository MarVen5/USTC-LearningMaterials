\documentclass{article}
\usepackage[margin=0.8in]{geometry}
\usepackage{amsmath}
\usepackage{ctex}
\usepackage{siunitx}
\usepackage{multirow}
\usepackage{bigstrut}
\usepackage{graphicx}
\usepackage{hyperref} 
\usepackage{mathbbol}

\title{切变模量实验报告}
\author{姓名:宋建宏\,\, 学号:PB21020677\,\, 班级:203院22级5班\\ 日期:2023年6月9日}
\date{}

\begin{document}
\maketitle

\section*{实验目的}
理解切变模量的物理意义,学习根据不确定度分析设计实验的方法。
\section*{实验原理}
对于一根金属丝,其形状类似于一个圆柱体,长度为$L$、半径为$ R$.固定一端,在材料弹
性限度内,扭转另一端,则圆柱体各截面的体积元均发生切应变,其切应变$\gamma$与切应力$\tau$比
值为常数:
\[G=\frac{\tau}{\gamma}\]
这个式子即为剪切胡克定律,$G$即为材料的切变模量。

在本实验中我们采用扭转金属丝的方式对切变模量进行测量。金属丝扭转时,
单位长度的转角满足 $\frac{\mathrm{d} \varphi} { \mathrm{d} L}=\frac{\varphi}{L} $, 分析这细圆柱中长为的一小段, 其上截面为 $\mathrm{A}$, 下截面 为 $B$。
发生切变时, 其下端 b 移动到 $b^{\prime}$, 其转角 $\gamma$ 满足
$b b^{\prime}=\gamma d l=R d \varphi$,
即切应变满足
$\gamma=R \frac{d \varphi}{d l}$,
这一小段钢丝内部离轴线距离为 $\mathrm{p}$ 的位置, 满足
$\gamma_\rho=\rho \frac{d \varphi}{d l}$,
此处的切应力为 $$\tau_\rho \cdot \rho \cdot 2 \pi \rho \cdot d \rho=2 \pi G \rho^3 \frac{d \varphi}{d l} \cdot d \rho$$, 因此对其积分得到恢复力矩:
$M=\int_0^R 2 \pi G \rho^3 d \rho \cdot \frac{d \varphi}{d l}=\frac{\pi}{2} G R^4 \frac{d \varphi}{d l}=\frac{\pi}{2} G R^4 \frac{\varphi}{l}$
因此有切变模量 $$G=\frac{2 D L}{\pi R^4}$$
设盘的转动惯量为 $\mathrm{I}_0$,得 $\frac{d^2 \varphi}{d t^2}+\frac{D}{I_0} \varphi=0$, 因此其
周期满足 $T_0=2 \pi \sqrt{\frac{I_0}{D}}$。由于扭摆底盘形状不规则, 因此可以通过尝试增加一个形状规则的辅助物体来间接得出 则其转动惯量之和为 $\mathrm{I}_0+\mathrm{I}_1$, 因此可以测出另一组周期 $\mathrm{T}_1$, 其满足 $T_1=2 \pi \sqrt{\frac{I_0+I_1}{D}}$, 因此可以 得到
$$
\begin{gathered}
D=\frac{4 \pi^2}{T_0^2} I_0=4 \pi^2 \frac{I_1}{T_1^2-T_0^2}=\frac{2 \pi^2 m\left(r_{\text {内 }}^2+r_{\text {外 }}^2\right)}{T_1^2-T_0^2} \\
G=\frac{4 \pi L m\left(r_{\text {內 }}^2+r_{\text {外 }}^2\right)}{R^4\left(T_1^2-T_0^2\right)}
\end{gathered}
$$
从而得到实验结果。

\section*{实验仪器}
切变模量实验仪器一套(铁架台,固定装置,金属丝,托盘,金属圆环),电子秒表,卷尺,游标卡尺,螺旋测微器。

\section*{数据处理}

\subsection*{金属丝的直径}
直径平均值\[\bar{D}=\frac{\sum_{i=1}^{10}D_i}{10}=\SI{0.7765}{mm}\]
A类标准不确定度\[\begin{aligned}
    u_A=\sqrt{\frac{\sum_{i=1}^{10} (D_i-\bar{D})^2}{10(10-1)}}=0.0012\,\mathrm{mm}
\end{aligned}\]
B类标准不确定度
\[u_B=\frac{\sqrt{\Delta_\text{仪}^2+\Delta_\text{估}^2}}{C}=\frac{\sqrt{0.004^2+0.005^2}}{3}=0.002\,\mathrm{mm}\]
展伸不确定度
$$
        U_{D} =\sqrt{\left(t_Pu_A\right)^2+\left(k_Pu_B\right)^2}=\sqrt{\left(2.26\times{0.0012}\right)^2+\left(1.96\times{0.002}\right)^2}=0.005\,\mathrm{mm}\quad(P=0.95)
$$

\subsection*{其他数据}
\subsubsection*{金属丝的长度}
由于测量时无法对准,估计误差约为$0.1\,\mathrm{cm}$,
B类标准不确定度
\[u_B=\frac{\sqrt{\Delta_\text{仪}^2+\Delta_\text{估}^2}}{C}=\frac{\sqrt{0.08^2+0.1^2}}{3}=0.04\,\mathrm{cm}\]
展伸不确定度 
$$
U_{L}=k_p u_B=1.96\times 0.04=0.08\,\mathrm{cm}  \quad(P=0.95)    
$$
\subsubsection*{金属环内外径}
展伸不确定度 
$$
U_{d}=k_p \frac{\sqrt{\Delta_\text{仪}^2+\Delta_\text{估}^2}}{C}=1.645\times \frac{\sqrt{0.02^2+0.01^2}}{\sqrt{3}}=0.02\,\mathrm{mm}  \quad(P=0.95)    
$$
\subsubsection*{扭摆周期}
周期的主要误差来源为人的估计误差,测量值的展伸不确定度为
\[U_t=k_p\frac{\Delta_{估}}{C}=1.96\times\frac{0.2}{3}=\SI{0.130}{s}\]
则周期测量值的不确定度为
\[U_{T_0}=\frac{U_t}{30}=\SI{0.004}{s}\]
\[U_{T_1}=\frac{U_t}{50}=\SI{0.003}{s}\]
\subsection*{最终计算}
代入公式计算得
\[D=\SI{5.570e-3}{N\cdot m}\qquad G=\SI{6.704e10}{Pa}\]
由不确定度传递公式
\[
\begin{aligned}
    \frac{U_G}{G} & =\sqrt{\left(\frac{U_L}{L}\right)^2+\left(\frac{U_m}{m}\right)^2+\left(\frac{4 U_D}{D}\right)^2+\frac{\left(2 d_{\text {内 }} U_{d}\right)^2+\left(2 d_{\text {外 }} U_{d}\right)^2}{\left(d_{\text {内 }}^2+d_{\text {外 }}^2\right)^2}+\frac{\left(2 T_0 U_{T_0}\right)^2+\left(2 T_1 U_{T_1}\right)^2}{\left(T_0^2-T_1^2\right)^2}} \\
    \frac{U_D}{D} & =\sqrt{\left(\frac{U_m}{m}\right)^2+\frac{\left(2 d_{\text {内 }} U_{d}\right)^2+\left(2 d_{\text {外 }} U_{d}\right)^2}{\left(d_{\text {内 }}^2+d_{\text {外 }}^2\right)^2}+\frac{\left(2 T_0 U_{T_0}\right)^2+\left(2 T_1 U_{T_1}\right)^2}{\left(T_0^2-T_1^2\right)^2}}
    \end{aligned}\]
已知$\frac{U_m}{m}=0.00018$,计算得
\[\frac{U_G}{G}=0.026 \qquad\frac{U_D}{D}=0.003\]
因此\[U_G=\SI{0.17e10}{Pa}\qquad U_D=\SI{0.017e-3}{N\cdot m}\]
故最终测量结果为\begin{align*}
    \text{切变模量:}& G=(6.70 \pm 0.17) \times 10^{10} \,\mathrm{P a}\quad(P=0.95)\\
\text{扭转模量:}& D=(5.570 \pm 0.017) \times 10^{-3}\, \mathrm{N} \cdot \mathrm{m}\quad(P=0.95)
\end{align*}

\section*{思考题}
\begin{enumerate}
    \item 本实验是否满足 $\gamma<<1$ 的条件?
    
    由于钢丝均匀形变, 有
$$
\gamma=R \frac{\varphi}{L}
$$
在实际实验中, 转动的角度 $\varphi<\pi$, 带入 $\mathrm{R}$ 和 $\mathrm{L}$ 的数据, 得 $\gamma<3 \times 10^{-3}$ 实际操作中, 转角更小, 可以满足 $\gamma<<1$ 的条件。

    \item 为提高测量精度,本实验在设计上作了哪些安排?在具体测量时又要注意什么?
    
    为提高实验精度,首要的就是减小主要误差。本实验通过仪器的最大允差,结合粗测
数据,从而选择了合适的仪器,并确定了金属丝的半(直)径的误差为实验的主要误差。因
此,对于其他物理量,测量1次即可;对于金属丝的直径则测量了10组以减小误差。考虑
到时间测量问题,需要根据要求,测量多个周期的总用时,从而减小这一项的实验误差。合
成以后,本实验的实验误差就可以降低。
\end{enumerate}


\end{document}