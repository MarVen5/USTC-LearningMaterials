\documentclass{article}
\usepackage[margin=0.8in]{geometry}
\usepackage{amsmath}
\usepackage{ctex}
\usepackage{siunitx}
\usepackage{multirow}
\usepackage{bigstrut}
\usepackage{graphicx}
\usepackage{hyperref} 
\usepackage{mathbbol}

\title{透镜参数测量实验报告}
\author{姓名:宋建宏\,\, 学号:PB21020677\,\, 班级:203院22级5班\\ 日期:2023年6月4日}
\date{}

\begin{document}
\maketitle

\section*{实验目的}
了解光源、物、像之间的关系以及球差、色差产生的原因,熟练掌握光具座上各种光学元件的调节并且初步学习光路设计,测量薄透镜的焦距。
\section*{实验原理}

\subsection*{平面镜反射法(自准直法)}
如图3位于焦点F上的物A所发出的光经过透镜变成平行光。再经平面镜M反射后可在物屏上得到清晰的倒立像A'。

\subsection*{公式法}

在近轴条件下高斯公式成立, 设 $p$ 为物距, $p^{\prime}$ 为像距, 物方焦距 (驰称前焦距) 为 $f$, 像方焦 距 (哇称后焦距 ) 为 $f$ 则有:
$$
\frac{f^{\prime}}{p^{\prime}}+\frac{f}{p}=1
$$
由于在空气中 $f=-f^{\prime}$, 高斯公式变成
$$
\frac{1}{p^{\prime}}-\frac{1}{p}=\frac{1}{f^{\prime}}
$$
\subsection*{位移法}
当物和屏之间的距离 $L$ 大于 $4 f$ 时, 固定物和屏, 移动透镜至 $C 、 D$ 处, 在像屏上可分别获得放大和缩小的实像。C、D 间距离为 $l$, 有
$$
f=\frac{L^2-l^2}{4 L}
$$
通过, 只要测得 $L $、$ l$, 即可获得焦距 $f$ 。


\section*{实验仪器}
光具座(包括光源、物屏、凸透镜、凹透镜、像屏等器具)。

\section*{数据处理}

\subsection*{凸透镜}

\subsubsection*{物像距法}
物距:$p=\SI{13.61}{cm}$
像距平均值\[p'=\frac{35.36+31.28+35.42+35.50+35.61+35.56}{6}=\SI{35.455}{cm}\]
由高斯公式得焦距\[f=\frac{13.61\times 35.455}{13.61+35.455}=9.834761\,\mathrm{cm}\]

物距的展伸不确定度只包含B类不确定度
\[U_p=u_B=k_p\frac{\sqrt{\Delta_\text{仪}^2+\Delta_\text{估}^2}}{C}=1.96\times \frac{\sqrt{0.12^2+0.05^2}}{3}=0.08\,\mathrm{cm}\quad (P=0.95)\]
像距的A类标准不确定度\[\begin{aligned}
    u_A&=\sqrt{\frac{(35.36-35.455)^2+(35.28-35.455)^2+(35.42-35.455)^2+(35.50-35.455)^2+(35.61-35.455)^2+(35.56-35.455)^2}{6(6-1)}}\\
    &=\SI{0.05}{cm}
\end{aligned}\]
B类标准不确定度
$$
    u_B=\frac{\sqrt{\Delta_\text{仪}^2+\Delta_\text{估}^2}}{C}=\frac{\sqrt{0.12^2+0.05^2}}{3}=0.04\,\mathrm{cm}
$$
展伸不确定度
$$
U_{p'}=\sqrt{\left(t_p u_A\right)^2+\left(k_pu_B\right)^2}=\sqrt{\left(2.57\times 0.05\right)^2+\left(1.96\times0.04\right)^2}=0.15\,\mathrm{cm}  \quad(P=0.95)    
$$
求得不确定度传递公式,得焦距不确定度
\[{U_f}=f\sqrt{\left(\frac{p'}{p(p+p')}U_p\right)^2+\left(\frac{p}{p'(p+p')}U_{p'}\right)^2}=0.04\,\mathrm{cm}\]
故最终结果为
\[f=(9.83\pm0.04)\,\mathrm{cm}\quad (P=0.95)\]

\subsubsection*{位移法}
物屏距$L=\SI{55.49}{cm}$ 
\[\bar{l}=\frac{41.98+41.95+41.94+42.01+41.96+41.98-13.66-13.63-13.60-13.69-13.62-13.72}{6}=28.317\,\mathrm{cm}\] 
测得焦距$$
f=\frac{L^2-l^2}{4 L}=\frac{55.48^2-28.317^2}{4\times55.48}=10.26\,\mathrm{cm}
$$
\subsubsection*{自准直法}
焦距\[f=\frac{9.78+9.75+9.83+9.81+9.77+9.79+10.22+10.21+10.29+10.26+10.23+10.18}{6}=10.01\,\mathrm{cm}\]


\subsection*{凹透镜}



\subsubsection*{物像距法}
物距平均值\[\bar{p}=\frac{28.19+28.06+28.12-3\times13.68}{3}=14.443\,\mathrm{cm}\]
像距平均值\[\bar{p'}=\frac{39.65+40.68+40.05}{3}=40.127\,\mathrm{cm}\]
由高斯公式得焦距
\[f=\frac{pp'}{p'-p}=\frac{14.443\times40.127}{40.127-14.443}=22.56\,\mathrm{cm}\]
\subsubsection*{自准直法}
凸透镜像距平均值为
\[\bar{q}=\frac{37.50+37.08+37.29}{3}=37.29\,\mathrm{cm}\]
凹凸镜距平均值为
\[\bar{l}=\frac{14.46+14.43+14.56+15.73+15.38+15.59}{6}=15.025\,\mathrm{cm}\]

得焦距为\[f=q-l=22.265\,\mathrm{cm}\]



\section*{思考题}
\begin{enumerate}
    \item 如果在“1”字屏后不加毛玻璃,对实验会有什么影响?
    
    由于光源发出的光本身很强,为使实验现象更加明显和使光线更加柔和均匀,故增加毛玻璃。而且毛玻璃粗糙一面恰有形成一个物点的作用。不加毛玻璃,在实验时难以确定是否对好焦。

    \item 自准直法测凸透镜焦距时,如果透镜安装在光具座上时沿光轴方向与光具座中心不重合(偏心),而我们测量距离时测量的是光具座之间的距离(默认为光学元件位于光具座中心位置),这对测量有什么影响?如何消除这一影响?
    
    使得测量光具距离时存在系统误差,可以选择在测量时翻转透镜取平均值。

    \item 在利用公式法和位移法测凸透镜焦距时,如果透镜安装时也存在上述偏心,对实验测量结果是否有影响?
    
    公式法有影响,位移法测量的是两个成像位置的距离,故偏心无影响。
\end{enumerate}


\end{document}