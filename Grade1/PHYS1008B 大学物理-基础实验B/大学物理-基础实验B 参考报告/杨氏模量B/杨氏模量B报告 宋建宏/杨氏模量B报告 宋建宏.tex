\documentclass{article}
\usepackage[margin=0.65in]{geometry}
\usepackage{amsmath}
\usepackage{ctex}
\usepackage{siunitx}
\usepackage{multirow}
\usepackage{bigstrut}
\usepackage{graphicx}
\usepackage{hyperref} 
\usepackage{mathbbol}

\title{杨氏模量实验报告}
\author{姓名:宋建宏\,\, 学号:PB21020677\,\, 班级:203院22级5班\\ 日期:2023年5月27日}
\date{}

\begin{document}
\maketitle

\section*{实验目的}
学习用拉伸法测定钢丝杨氏模量的方法,掌握利用光杠杆测定微小形变的方法。

\section*{实验原理}
在材料弹性限度内, 应力 $F / S$ (即法向力与材料横截面积之比) 和应变 $\Delta L / L$ (即长度的相 对延长量)之比是一个常数, 即
$$
    E=\frac{F / S}{\Delta L / L}=\frac{FL}{S \Delta L}
$$
为常数。这个数值叫做杨氏模量。

由于金属丝杨氏模量大, 施加常规大小的应力时, 产生的 $\triangle \mathrm{L}$ 很小, 不易被测量。因此需要对 $\triangle \mathrm{L}$ 采用放大的方法, 比如光杠杆放大法。

当金属丝受到向下拉力 $F$ 作用时, 杠杆支脚将随被测物下降微小距离 $\triangle L$, 平面镜镜面的法线将转过一个 $\theta$ 角, 此时从望远镜中看到的标尺刻度是标尺经过平 面镜反射所成的像, 从尺子发出的入射线和反射线的夹角为 $2 \theta$,  当 $\theta$ 很小时,
$$
    \tan 2 \theta \approx 2 \theta=\frac{b}{D}\quad\quad \theta \approx \tan \theta =\frac{\Delta L}{l}
$$
式中 $D$ 为镜面到标尺的距离, $\mathrm{b}$ 为在拉力 $F$ 作用下标尺读数的改变。
由上式可得
$$
    \frac{\Delta L}{l}=\frac{b}{2 D}
$$
由此得
$$
    \Delta L=\frac{b l}{2 D}
$$
和
$$
    E=\frac{2 D L F}{S l b}
$$
式中 $2 D / l$ 叫做光杠杆的放大倍数。测出 $D 、 L 、 l$ 和金属丝直径 $d\left(S=\pi d^2 / 4\right)$ 及一系列的 $F$ 与 $b$ 之后, 即可计算出金属丝的杨氏模量 $E$ 。
\begin{figure*}[htbp]
    \centering
    \includegraphics*[scale=0.7]{figure.png}
\end{figure*}

\section*{实验仪器}
杨氏模量实验仪器一套(金属丝,支架,砝码盘,砝码,标尺,望远镜)。

\section*{测量记录}
原始数据见附纸。
\begin{table*}[htbp]
    \centering
    \caption{金属丝的直径d,螺旋测微器的初始读数 $d_0=\SI{-0.010}{mm}$}
    \begin{tabular}{|l|l|l|l|l|l|l|}
        \hline
        测量序号          & 1     & 2     & 3     & 4     & 5     & 6    \bigstrut  \\
        \hline
        读数            & 0.284 & 0.286 & 0.285 & 0.283 & 0.287 & 0.283 \bigstrut \\
        \hline
        $d/\SI{}{mm}$ & 0.294 & 0.296 & 0.295 & 0.293 & 0.297 & 0.293 \bigstrut \\
        \hline
    \end{tabular}
\end{table*}

\begin{table*}[htbp]
    \centering
    \caption{砝码个数与标尺读数 b 的关系}
    \begin{tabular}{|cc|r|r|r|r|r|r|r|r|}
        \hline
        \multicolumn{2}{|c|}{砝码个数}                             & 0                       & 1 & 2 & 3 & 4 & 5 & 6 & 7 \bigstrut             \\
        \hline
        \multicolumn{1}{|r|}{\multirow{2}[4]{*}{读数/\SI{}{mm}}} & \multicolumn{1}{l|}{去程} &0.00   &1.53   &3.09   & 4.71  & 6.23  & 7.81  &  9.49           &11.09 \bigstrut \\
        \cline{2-10}    \multicolumn{1}{|r|}{}                 & \multicolumn{1}{l|}{回程} & 0.09  & 1.61  &  3.14 & 4.78  & 6.43  & 7.97  &       9.56      & 11.09\bigstrut \\
        \hline
        \multicolumn{2}{|c|}{b/\SI{}{mm}}                                                   & 0.045 & 1.57 &  3.115 &  4.745 &  6.33 & 7.89 & 9.525 &11.09 \bigstrut               \\
        \hline
    \end{tabular}
\end{table*}

\section*{数据处理}
\subsection*{标尺到平面镜的距离D}
$$
    \bar{D}=\frac{155.53+155.48 +155.51 }{3}\,\mathrm{cm}=155.507\,\mathrm{cm}
$$
标准差
$$
    \begin{aligned}
        \sigma=\sqrt{\frac{(155.53-155.507)^2+(155.48-155.507)^2+(155.51-155.507)^2}{3-1}}=0.025\,\mathrm{cm}
    \end{aligned}
$$
B类极限不确定度
$$
    \Delta=\sqrt{\Delta_\text{仪}^2+\Delta_\text{估}^2}=\sqrt{0.12^2+0.05^2}=0.13\,\mathrm{cm}
$$
合成不确定度
$$
U_{D}=\sqrt{\left(t_P\frac{\sigma}{\sqrt{3}}\right)^2+\left(k_P\frac{\Delta}{C}\right)^2}=\sqrt{\left(4.3\times \frac{0.025}{\sqrt{3}}\right)^2+\left(1.96\times\frac{0.13}{3}\right)^2}=0.1\,\mathrm{cm}  \quad(P=0.95)    
$$

\subsection*{光杠杆的臂长l}
$$
    \bar{l}=\frac{7.16+7.15+7.14}{3}\,\mathrm{cm}=7.15\,\mathrm{cm}
$$
标准差
$$
    \begin{aligned}
        \sigma=\sqrt{\frac{(7.16-7.15)^2+(7.15-7.15)^2+(7.14-7.15)^2}{3-1}}=\SI{0.01}{cm}
    \end{aligned}
$$
B类极限不确定度
$$
    \Delta=\sqrt{\Delta_\text{仪}^2+\Delta_\text{估}^2}=\sqrt{0.12^2+0.05^2}\,\mathrm{cm}=0.13\,\mathrm{cm}
$$
合成不确定度
$$
U_l=\sqrt{\left(t_P\frac{\sigma}{\sqrt{3}}\right)^2+\left(k_P\frac{\Delta}{C}\right)^2}=\sqrt{\left(4.3\times \frac{0.01}{\sqrt{3}}\right)^2+\left(1.96\times\frac{0.13}{3}\right)^2}=0.09\,\mathrm{cm}  \quad(P=0.95) 
$$

\subsection*{钢丝原长L}
$$
    \bar{L}=\frac{105.86+105.83+105.84}{3}\,\mathrm{cm}=105.843\,\mathrm{cm}
$$
标准差
$$
    \begin{aligned}
        \sigma=\sqrt{\frac{(105.86-105.843)^2+(105.83-105.843)^2+(105.84-105.843)^2}{3-1}}=\SI{0.015}{cm}
    \end{aligned}
$$
B类极限不确定度(因测量时无法对准,估计误差约为\SI{0.5}{cm})
$$
    \Delta=\sqrt{\Delta_\text{仪}^2+\Delta_\text{估}^2}=\sqrt{0.12^2+0.5^2}\,\mathrm{cm}=0.5\,\mathrm{cm}
$$
合成不确定度
$$
U_L=\sqrt{\left(t_P\frac{\sigma}{\sqrt{3}}\right)^2+\left(k_P\frac{\Delta}{C}\right)^2}=\sqrt{\left(4.3\times \frac{0.015}{\sqrt{3}}\right)^2+\left(1.96\times\frac{0.5}{3}\right)^2}=0.3\,\mathrm{cm}  \quad(P=0.95) 
$$
\subsection*{钢丝直径d}
$$
    \overline{d}=\frac{0.294+0.296+0.295+0.293+0.297+0.293}{6}\,\mathrm{mm}=0.2947\,\mathrm{mm}
$$
标准差
$$
    \begin{aligned}
        \sigma  & =\sqrt{\frac{(0.294-0.2947)^2+(0.296-0.2947)^2+(0.295-0.2947)^2+(0.293-0.2947)^2+(0.297-0.2947)^2+(0.293-0.2947)^2}{6-1}}\\
                   & =0.0012\,\mathrm{mm}
    \end{aligned}
$$
B类极限不确定度
$$
    \Delta_{B,d}=\sqrt{\Delta_\text{仪}^2+\Delta_\text{估}^2}=\sqrt{0.004^2+0.005^2}\,\mathrm{mm}=0.006\,\mathrm{mm}
$$
展伸不确定度
$$
        U_{d,P}  =\sqrt{\left(t_P\frac{\sigma}{\sqrt{6}}\right)^2+\left(k_P\frac{\Delta_{B,d}}{C}\right)^2}            =\sqrt{\left(2.57\times\frac{0.0012}{\sqrt{6}}\right)^2+\left(1.96\times\frac{0.006}{3}\right)^2}=0.004\,\mathrm{mm}\quad(P=0.95)
$$

\subsection*{砝码数与标尺读数b的关系}

最小二乘法拟合如图
\begin{figure*}[htbp]
    \centering
    \includegraphics*[scale=0.6]{graph0.png}
\end{figure*}

斜率
$    a=1.58333\,\mathrm{cm}
$,
线性拟合的相关系数
$
    r=0.99997
$,
因此斜率的展伸不确定度为
$$
    U_{\mathrm{a}}=t_p\sqrt{\frac{r^{-2}-1}{8-2}} a =2.45\times \sqrt{\frac{0.99997^{-2}-1}{8-2}}\times 1.58333 =0.012 \,\mathrm{cm} (P=0.95)
$$
因此 $a=(1.583 \pm 0.012) \mathrm{cm}(P=0.95)$

由上得杨氏模量为
$$
    E=\frac{8 D L \mathrm{mg}}{\pi \mathrm{d}^2 l} \cdot \frac{\Delta n}{\Delta b}=\frac{8 D L \mathrm{mg}}{\pi \mathrm{d}^2 l \mathrm{a}}=2.08836 \times 10^{11} \mathrm{~Pa}
$$
由于
$$
    \frac{\Delta E}{E}=\sqrt{\left(\frac{\Delta L}{L}\right)^2+\left(\frac{\Delta D}{D}\right)^2+\left(2 \frac{\Delta \mathrm{d}}{\mathrm{d}}\right)^2+\left(\frac{\Delta l}{l}\right)^2+\left(\frac{\Delta a}{a}\right)^2}=0.03
$$
故
\[\Delta E= 0.03E=\SI{0.06e11}{Pa}\]
最终结果为
$$
    E=\left(2.09 \pm 0.06\right) \times 10^{11}\,\mathrm{N/m^2}\quad (P=0.95)
$$



\section*{思考题}

\begin{enumerate}
    \item 利用光杠杆把测微小长度$\Delta l$变成测$ b$,光杠杆的放大率为$ 2D/L$,根据
          此式能否以增加$D$减小$l$来提高放大率,这样做有无好处?有无限度?应怎样考虑这个问题?

          作用不大,应该合理地提高 D,不建议缩小l。本实验中,放大率已经足够,因此b的测量误差并不大。虽然 b 随之增加,但是由于
          本实验误差本身就比较大,在本实验情况下,b 已经不是主要误差。过分追求放大b对于实
          验精度贡献小。当 l 较小时,可能引发偏转角$\theta$过大,从而引入新的误差$\tan \theta =\theta$。
          l 的减小容易使得其相对误差较大,使实验误差大。另外,D 的增大,使标尺的像
          难以被找到,同时也会减弱像的稳定性,增加实验复杂度。
          当放大率大时,很有可能使操作过程中标尺读数变化过大,从而超出量程,因此不得
          不重新实验,或者换用更长的标尺,操作复杂。
          因此,实验中应保持合适的D与l,从而提升精度、简化步骤。

    \item 实验中,各个长度量用不同的仪器来测量是怎样考虑的,为什么?

          主要从被测物和仪器的特性以及误差均分原理考虑。对于被测物来说,其大致长度
          决定了仪器需要的量程;实际情况则限制了测量方法,进而限制了仪器的种类。由误差均分
          原理得到,对于不同的被测物,其需要的绝对误差不同。因此对于仪器的精度要求不同。
          综合考虑以上几点,可以选择合适的仪器,提升实验精度,提升实验效率。
\end{enumerate}

\end{document}