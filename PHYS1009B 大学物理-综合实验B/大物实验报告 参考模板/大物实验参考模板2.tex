\documentclass[UTF8]{ctexart}
\usepackage{ntheorem}
\usepackage{amsmath}
\usepackage{upgreek}
\usepackage{graphicx}
\usepackage{footmisc}
\usepackage{amssymb}
\usepackage{makecell}
\usepackage{wrapfig}
\usepackage{geometry}

\title{表面张力实验数据处理}
\author{马文宇 PB23061139 }
\geometry{a4paper,scale=0.75}
\begin{document}
\maketitle
实验时间:2024.5.29晚上

实验教室:1423

座位号:8

\section{作图法求锥形弹簧的弹性系数}

\begin{figure}[!htp]
    \centering
    \includegraphics[scale=0.3]{D:/photos/651.png}
    \caption{质量与距离的线性拟合图}
\end{figure}

根据公式$mg=kx$,以及合肥地区的重力加速度$g\approx9.795m/s^2$,可知锥形弹簧的弹性系数$k=1.1708N/m$。

\section{自来水的表面张力系数}

金属圈直径$d=\frac{1}{3}(2.97+3.00+3.02)cm=2.9967cm$,金属圈周长$s=\pi d=9.4143cm$。

初始距离$l_0=0.51cm$,液膜破裂距离$l=\frac{1}{5}(1.42+1.48+1.46+1.47+1.48)cm=1.462cm$。

根据$k(l-l_0)=2\sigma s$,得到水的表面张力系数$\sigma_1=0.0592N/m$

\section{洗洁精的表面张力系数}

金属丝长度$s=\frac{1}{3}(4.21+4.20+4.20)cm=4.2067cm$。

初始距离$l_0=0.86cm$,液膜破裂距离$l=\frac{1}{5}(1.07+1.04+1.05+1.05+1.04)cm=1.050cm$。

根据$k(l-l_0)=2\sigma s$,得到洗洁精的表面张力系数$\sigma_2=0.0264N/m$

\section{三个浓度表面张力系数关系曲线}

初始距离$l_0=0.82cm$,

当体积比为$1:150$时,液膜破裂距离$l=1.05cm$;

当体积比为$1:300$时,液膜破裂距离$l=1.04cm$;

当体积比为$1:500$时,液膜破裂距离$l=1.03cm$。

\begin{figure}[!htp]
    \centering
    \includegraphics[scale=0.3]{D:/photos/652.png}
    \caption{浓度表面张力系数关系曲线}
\end{figure}

\end{document}